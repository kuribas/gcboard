\documentclass{article}

% Packages required to support encoding
\usepackage{ucs}
\usepackage[utf8x]{inputenc}

% Packages required by code


% Packages always used
\usepackage{hyperref}
\usepackage{xspace}
\usepackage[usenames,dvipsnames]{color}
\hypersetup{colorlinks=true,urlcolor=blue}



\begin{document} 
\hypertarget{about}{}\subsection*{{About}}\label{about}

\textbf{gcboard (Generic Chess Board)} is a gameboard API for Chess-style games (i.e. \emph{Draughts, Shogi, Xiangqi, Checkers, Chess,} and variants). It implements most of the features that are usually found in chess interfaces (dragging, highlighting, animating, flashing, etc.) and handles them without additional effort for the programmer. Gcboard will support your game if it has the following properties:

\begin{itemize}%
\item The game is turn based (from 2 to 15 players).
\item Pieces are put on one or more rectangular grids of squares (typically one).
\item Moves are made by moving a piece from it'{}s square to another (either empty or containing an enemy piece).

\end{itemize}
The following games don'{}t fall into this cathegory: \emph{go, othello, monopoly, backgammon.}

\emph{GCBoard} is \textbf{platform independent}. The core of the library has been written in C with portability in mind. Supporting new platforms and toolkits should be very easy: just create a wrapper around the platform independent code. Currently there is support for the \href{"http://www.gtk.org"}{Gtk2} toolkit. Any contributions to support other toolkits and platforms are certainly welkom! (e.g. MS Windows, Qt, Mac Os).

The features that are more cpu-intensive (e.g. animating pieces) are already implemented in the core, so that makes it possible to use this library with languages that have slow interpreters, such as \emph{ruby,} \emph{scheme, python and perl}. Bindings exist for \href{"http://www.ruby-lang.org"}{Ruby}, and I would gladly accept contributions for other bindings.

I hope that this library will encourge other people to write beautiful games, without having to spend to much effort on writing the gui. Instead of having to write a full interface each time from scratch, \emph{GCBoard} makes it possible to concentrate on the game-specific parts.

\emph{GCBoard} is released under the \href{"http://www.gnu.org/copyleft/gpl.html"}{General Public Licence}. Basicly this means that you are free to use and modify this library as long as you also release the sources to your program. (\textbf{Note:} this is not the LGPL license, it doesn'{}t permit linking (either staticaly or dynamicaly) with closed source programs.

I'{}d like to hear your opinion about gcboard, or requests of new features. If you know a feature or enhancement that is generally usefull I will probably add it.

Gcboard is developed by Kristof Bastiaensen and hosted by \href{"http://sourceforge.net"}{sourceforge}. You can find the project page \href{"http://sourceforge.net/projects/gcboard"}{here}.

\vfill
\hrule
\vspace{1.2mm}
\begin{tiny}
Created by \href{http://maruku.rubyforge.org}{Maruku}  at 22:54 on Wednesday, May 09th, 2007.
\end{tiny}
\end{document}
